% fecha: 01/04/2015
% autor: Víctor Peinado @vitojph

% declaramos el tipo de documento
\documentclass[a4paper]{article}

% en la cabecera, añadimos los paquetes extra

% imprescindibles
\usepackage[spanish]{babel} % definimos el español como lengua base
\usepackage[utf8]{inputenc} % definimos UTF8 como codificación

% opcionales
\usepackage{multirow} % fusión filas en tablas 
\usepackage{hyperref} % introduce enlaces a partir de URLs

\usepackage{tipa} % paquetes para símbolos del IPA
\usepackage{tipx}

\usepackage{qtree} % árboles sintácticos
\usepackage{tikz-dependency} % árboles de dependencias

\usepackage{gb4e} % glosas


% definimos el título, la fecha, etc
\title{Introducción a \LaTeX}
\author{Víctor Peinado}
\date{1 de abril de 2015}


% aquí termina la cabecera y comienza el cuerpo del documento
\begin{document}

% pintamos el título
\maketitle

% comienza el contenido de nuestro documento 
\section{Introducción}
\label{sec:intro}

En este documento, presentamos los comandos básicos para crear documentos en \LaTeX: introducir (Sección \ref{parrafos})  y formatear texto (Sección \ref{estilos}), organizar la información en listas (Sección \ref{listas}) y tablas (Sección \ref{tablas}), etc. 

En la última parte, mostraremos algunos ejemplos de paquetes de \LaTeX\ útiles para lingüistas (Sección \ref{ling}).

Lo interesante es leer el código y compararlo con la salida en PDF. Así que , para que esto realmente sea útil, usa ambos ficheros a la vez.


\section{Párrafos}
\label{parrafos}

Para organizar el texto en párrafos no es necesario que los marquemos de manera explícita: basta con que los separemos con (al menos) dos pulsaciones de intro o retornos de carro.

Aquí comienza mi segundo párrafo.                           
Aquí hay otra frase.                      Y aquí termina.

% fíjate en que es indiferente el número de espacios que introduzcas



Habrás notado que los párrafos se sangran automáticamente. Pero este comportamiento se puede modificar, con la instrucción \texttt{\textbackslash noindent}, como hago a continuación.

\noindent Este párrafo contiene unas cuantas palabras.


\noindent Este párrafo contiene unas cuantas palabras. Este párrafo contiene unas cuantas palabras. Este párrafo contiene unas cuantas palabras. Este párrafo contiene unas cuantas palabras.


\section{Estilos de letra y tamaños}
\label{estilos}

Para cambiar el estilo del texto, podemos utilizar distintos comandos:

\begin{itemize}
\item el comando \texttt{\textbackslash textit} para el \textit{texto en cursiva}.

\item el comando \texttt{\textbackslash textbf} para el \textbf{texto en negrita}.

\item el comando \texttt{\textbackslash textsc} para el \textsc{Texto En Versalita}.

\item el comando \texttt{\textbackslash texttt} para el \texttt{texto en fuente de ancho fijo}.

\end{itemize}


Con respecto al tamaño de letra, podemos manejar varios: desde el \tiny{diminuto}, \scriptsize{tamaño script}, \footnotesize{tamaño de pie de página}, \small{pequeño}, tamaño normal, \large{grande}, \Large{más grande}, \LARGE{todavía más grande}, \huge{enorme}, \Huge{gigantesco}.

\normalsize{Volvemos al texto normal.}



\section{Alineación del texto}
\label{alineacion}

Para justificar el texto, podemos definir entornos de distinto tipo con los comandos \texttt{\textbackslash begin} y \texttt{\textbackslash end}, a saber:

\begin{center}
Este párrafo está centrado con el entorno \texttt{center}. Bla, bla, bla, bla, bla, bla, bla, bla, bla, bla, bla, bla, bla, bla, bla, bla, bla, bla, bla, bla, bla, bla, bla, bla, bla, bla, bla, bla, bla, bla, bla, bla, bla, bla, bla, bla.
\end{center}


\begin{flushright}
Este párrafo está justificado a la derecha con el entorno \texttt{flushright}. Bla, bla, bla, bla, bla, bla, bla, bla, bla, bla, bla, bla, bla, bla, bla, bla, bla, bla, bla, bla, bla, bla, bla, bla, bla, bla, bla, bla, bla, bla, bla, bla, bla, bla, bla, bla.
\end{flushright}


\begin{flushleft}
Este párrafo está alineado a la izquierda y justificado por ambos lados (por defecto es así) con el entorno \texttt{flushleft}. Bla, bla, bla, bla, bla, bla, bla, bla, bla, bla, bla, bla, bla, bla, bla, bla, bla, bla, bla, bla, bla, bla, bla, bla, bla, bla, bla, bla, bla, bla, bla, bla, bla, bla, bla, bla.
\end{flushleft}


\section{Listas}
\label{listas}

En las sección\ref{estilos} hemos visto un par de ejemplos de listas. Entremos en detalle.

Para definir listas, podemos utilizar los comandos \texttt{\textbackslash begin} y \texttt{\textbackslash end} con las opciones siguientes:

\texttt{itemize} crea listas no numeradas:

\begin{itemize}
\item Primer elemento.
\item Segundo elemento.
\item Tercer elemento.
\end{itemize}

\texttt{enumerate} crea listas numeradas:

\begin{enumerate}
\item Primer elemento.
\item Segundo elemento.
\item Tercer elemento.
\end{enumerate}


\texttt{description} crea listas de definición:

\begin{enumerate}
\item[Primer elemento] es el que va antes del resto.
\item[Segundo elemento] no comienza ni termina la serie.
\item[Tercer elemento] es el último.
\end{enumerate}



Y recuerda que las listas se pueden anidar de distintas maneras:

\begin{enumerate}

\item tomates
  \begin{enumerate}
  	\item cherry
  	\begin{enumerate}
  		\item maduritos
  	\end{enumerate}
  	\item pera 
  	\item en rama
  \end{enumerate}

\item lechugas

\item patatas

\item pimientos

\end{enumerate}



\section{Tablas}
\label{tablas}

Un ejemplo de tabla sencilla, en la que no fusiono celdas pero sí aplico distintos estilos y alineación al texto de las celdas.

\vspace{0.5cm} % introduce un espacio vertical extra de 0,5 cm

\begin{tabular}{|| l | c | l | l | l ||}
    \hline
    \hline
   uno & dos & \textbf{tres} & xx & qq \\
    \hline
   cuatro & cinco \textit{cinq five cinque} & seis  & xx & qq \\
    \hline
   siete & ocho & nueve & xx & qq \\ 
    \hline
   \texttt{diez} & once & doce  & xx & qq \\
    \hline
    \hline
 \end{tabular}
 
\vspace{0.5cm} % introduce un espacio vertical extra de 0,5 cm

Otro ejemplo de tabla. En este caso sí fusiono columnas (en horizontal) y filas (en vertical). Para poder hacer esto último, recuerda que necesitarás cargar el paquete \texttt{multirow}.


\vspace{0.5cm} % introduce un espacio vertical extra de 0,5 cm


\begin{tabular}{| c | c | c |}
    \hline
    1 & 2 & 3 \\
    \hline
    
    \multirow{2}{*}{10} & 20 & 30 \\
    \cline{2-3} % pinta una línea horizontal de la col 2 a la 3
    
    & \multicolumn{2}{r|}{la suma total de los números anteriores es}\\

    \hline
    
    Una palabra & \multicolumn{2}{c|}{la suma total de los números anteriores es}\\

    \hline

    11 & 21 & 31\\
    \hline

\end{tabular}

\vspace{0.5cm} % introduce un espacio vertical extra de 0,5 cm

Sobre tablas hay muchísima información en la web, curiosea.\footnote{Revisa 
\url{http://es.wikibooks.org/wiki/Manual_de_LaTeX/Escribiendo_texto/Tablas} si quieres aprender más.}

\section{Paquetes de lingüística}
\label{ling}


\subsection{Símbolos del IPA}
\label{ipa}

\begin{center}
\textipa{abcdefghijklmnopqrstuvwxyz ABCDEFGHIJKLMNOPQRSTUVWXYZ 1234567890 @ \;A \;B \;E \;G \;H \;I \;L \;R \;Y \:d \:l \:n \:r \:s \:t \:z \!b \!d \!g \!j \!G \!o}
\end{center}


Quizá el paquete de \LaTeX\ de língüística más interesante es \texttt{tipa}, que nos permite introducir símbolos del \textit{International Phonetic Alphabet} (IPA) en nuestor documento. Veameos algunos ejemplos.

Una vez importados los paquetes \texttt{tipa} y \texttt{tipx} con la instrucción \texttt{usepackage} podremos introducir símbolos fonéticos de varias maneras: 

De uno en uno: en español \textipa{[b]} no es lo mismo que \textipa{[B]}, de hecho son alófonos en distribución complementaria.


Transcripciones fonéticas más largas:

\begin{IPA}
\noindent [pero si me "d\~an a elexír, \~e\c{n}tre tóDas las BíDas, \textdyoghlig o eskóxo la Del piráta kóxo k\~om páta De pálo, k\~om pár\textteshlig e \~en el óxo, k\~oN kára De málo, el Bjéxo tru"\~aN kapi"t\~a\c{n} de \textsubarch{\~u}m bárko ke tuBjéra por B\~a\c{n}déra "\~um pár De tíBjas j\textsubarch{ú}na kalaBéra.]
\end{IPA}


\subsection{Árboles sintácticos}
\label{arboles}

Ejemplos de árboles sintácticos en una lengua inventada.

\vspace{0.5cm}

\qtreecenterfalse
\hskip 0.1in
a. \Tree [.O [.SN [.Pron uk ] [.Det naam ] ] [.SV [.SN [.Pron ek ] [.Det kal ] ] [.V remit ] ] ]
\hskip 0.1in
b. \Tree [.O [.SN [.Adj saka ] [.N hollum ] [.Det naam ] ] [.SV [.V vomit ] ] ]
\hskip 0.1in
c. \Tree [.O [.SN [.N faros ] [.Det dar ] ] [.SV [.SP [.Pre taj ] [.SN [.Adj jelo ] [.N drom ] [.Det kal ] ] ] [.V homit ] ] ]


\subsection{Árboles de dependencias}
\label{deps}

Ejemplo sencillo de árbol de dependencias.


\begin{dependency}[theme = simple]
   \begin{deptext}[column sep=1em]
      El \& niño \& está \& llorando \& en \& el \& banco \&. \\
   \end{deptext}
   \deproot{4}{ROOT}
   \depedge{2}{1}{det}
   \depedge{4}{3}{aux}
   \depedge{4}{2}{nsubj}
   \depedge{4}{5}{prep}
   \depedge{7}{6}{det}
   \depedge{5}{7}{pobj}
   \depedge{4}{8}{p}
\end{dependency}


\subsection{Glosas}
\label{glosas}

Un par de ejemplos numerados con glosas, que nos permiten comparar la morfosintaxis de dos lenguas diferentes.

\begin{exe}
\ex
\gll Den Fritz_1 habe ich \_\_{}_1 zum Essen eingeladen.\\
the fred have I {} {to the} eating invited.\\
\glt I invited Fred for dinner.
\end{exe}

\begin{exe}\let\eachwordtwo=\sf
\ex[*]{
\gll Cette phrase n'a pas une traduction.\\
This sentence {not has} nought a translation.\\}
\end{exe}


\end{document}
% fin del documento
